\documentclass[handout,8pt]{beamer} % handout : désactive les pauses

\usetheme[compress]{Dresden}
%\setbeamertemplate{page number in head/foot}[framenumber]
\addtobeamertemplate{navigation symbols}{}{ 
	\hspace{1em}    
	\usebeamerfont{footline}%
    \insertframenumber / \inserttotalframenumber 
}
    
\usepackage[utf8]{inputenc}
\usepackage[french]{babel}
\usepackage[T1]{fontenc}
\usepackage{amsmath}
\usepackage{amsfonts}
\usepackage{amssymb}
\usepackage{graphicx}

\usepackage{hyperref}
\usepackage{multicol}
\usepackage{tikz}
\usetikzlibrary{3d}

\uselanguage{French}

\languagepath{French}



%%%%%%%%%%%%%%%%%%%% Page de garde %%%%%%%%%%%%%%%%%%%%

\title{La Physique des 2 Infinis}
\author[A. Hocine]{Alexia HOCINE\\ \textit{Physicienne subatomique \& Développeuse}}

%\setbeamercovered{transparent} 
%\setbeamertemplate{navigation symbols}{} 
%\logo{} 

\institute[LVA]{Les Vendredis de l'Astronomie} 

\date[2023]{Vendredi 1\ier\ Décembre 2023} 

%%%%%%%%%%%%%%%%%%%% Environnement %%%%%%%%%%%%%%%%%%%%

% definition

\begin{document}

%%%%%%%%%%%%%%%%%%%% Page de garde %%%%%%%%%%%%%%%%%%%%

\begin{frame}
	\titlepage
\end{frame}

%%%%%%%%%%%%%%%%%%%% Introduction %%%%%%%%%%%%%%%%%%%%

% Problématiques
\begin{frame}{Introduction}{Objectifs de cette conférence}
	\begin{center}
		\begin{Large}
			De quoi est constituée l'Univers ?
		\end{Large}
	\end{center}
	
	\begin{enumerate}
		\pause
		\item Quelles sont les constituants élémentaires ? Quelles sont leurs interactions ?
		\pause
		\item Quand ? Où ? Comment se sont-elles formées ?
		\pause
		\item Les connaissont-toustes ? Si non comment le sait-on ?
	\end{enumerate}
\end{frame}

% Plan général
\begin{frame}{Introduction}{Sommaire}
	\begin{center}
		\begin{Large}
			\textbf{La Physique des 2 Infinis}
		\end{Large}
	\end{center}

	\tableofcontents[hideallsubsections] % pausesections
\end{frame}

%%%%%%%%%%%%%%%%%%%% L'infiniment petit %%%%%%%%%%%%%%%%%%%%

\section{Vers l’infiniment petit}
\begin{frame}{Vers l’infiniment petit}{Sommaire}
	\begin{center}
		\begin{Large}
			\textbf{La Physique des 2 Infinis}
		\end{Large}
	\end{center}
    \tableofcontents[currentsection, hideallsubsections]
\end{frame}

\subsection{La Matière}
\begin{frame}{Qu'est-ce que la Matière ?}

\begin{definition}[La Matière \cite{matter}]
	\begin{itemize}
		\item ce qui compose tout corps,
		\item objet qui occupe de l'espace et qui ont une masse. 
	\end{itemize}
	Réciproquement, tout ce qui a une masse est de la matière.
\end{definition}
	
	\pause
	\begin{center}
		\textbf{De quoi est composé la matière ?}
	\end{center}
\end{frame}

\subsection{L’atome}
\begin{frame}{De quoi est composée la Matière ?}

\begin{columns}
	\begin{column}{0.7\textwidth}
	
		\begin{block}{Intuition de Démocrite (460-370 avant J.-C.)\cite{democrite}}
			La matière est composée d'objets élémentaires \textit{indivisibles}. 
		\end{block}
		\pause

		\begin{block}{Aristote, Nouveau Testament, ... }
			\begin{itemize}
				\item Nommé "\textbf{atome}" (partie de matière indivisible)
			\end{itemize}
		\end{block}

		\begin{definition}[L'Atome\cite{atom}]
			\begin{itemize}
				\item Petit corps
				\item La plus petite partie d'un corps simple pouvant se combiner chimiquement avec un autre. 
				\item Les constituants élémentaires de toutes les substances solides, liquides ou gazeuses.
			\end{itemize}
		\end{definition}

	\end{column}
	
	\begin{column}{0.25\textwidth}
		\begin{figure}
			\center
			\begin{tikzpicture}[scale=0.5]
				\shade[ball color = gray!40, opacity = 0.4] (0,0) circle (2cm);
  				\draw (0,0) circle (2cm);
  				\draw (-2,0) arc (180:360:2 and 0.6);
  				\draw[dashed] (2,0) arc (0:180:2 and 0.6);
			\end{tikzpicture}
			\caption{1\iere\ vision de l'atome}
		\end{figure}
		
		\begin{figure}
			\begin{tikzpicture}[scale=0.5]
				\shade[ball color = red!80, opacity = 0.4] (0,0) circle (1cm);
  				\draw (0,0) circle (1cm);
  				
  				\shade[ball color = gray!80, opacity = 0.4] (1.7,-1.7) circle (0.5cm);
  				\draw (1.7,-1.7) circle (0.5cm);
  				%\draw[dashed] (0,0) -- (2,-2);
  				\draw (0.7,-0.7) -- (1.3,-1.3);
  				
  				\shade[ball color = gray!80, opacity = 0.4] (-1.7,-1.7) circle (0.5cm);
  				\draw (-1.7,-1.7) circle (0.5cm);
				\draw(-0.7,-0.7) -- (-1.3,-1.3);
			\end{tikzpicture}
			\caption{Représentation d'une molécule d'eau}
		\end{figure}
	\end{column}
\end{columns}

\end{frame}

\subsubsection{L'électron}
\begin{frame}{Le modèle atomique de Thomson}

	\begin{columns}
		\begin{column}{0.7\textwidth}
		
			\begin{block}{Historique}
				\begin{description}
					\item[1838-1851] Prédiction de l'électron par Richard Laming 
					% dans une série d'article
					\item[1897] Découverte de l'électron par Joseph John Thomson
				\end{description}
			\end{block}

			\begin{block}{Propriétés}
				\begin{description}
					\item[1838-1851] Prédiction de l'électron par Richard Laming 
					% dans une série d'article
					\item[1897] Découverte de l'électron par Joseph John Thomson
				\end{description}
			\end{block}

			\begin{block}{Modèle atomique de Thomson}
				\begin{description}
					\item[1904] L'atome composé de charges positives et de particules négatives.
				\end{description}
			\end{block}

		\end{column}
		\begin{column}{0.25\textwidth}
		
			\begin{figure}
				\center
				\begin{tikzpicture}[scale=0.4]
					\def\r1{4};
					\shade[ball color = red, opacity = 0.5] (0,0) circle (\r1);
  					\draw (0,0) circle (\r1);
  					\draw (-\r1,0) arc (180:360:\r1 and 0.6);
  					\draw[dashed] (\r1,0) arc (0:180:\r1 and 0.6);
  					\filldraw[red,  opacity = 0.5] (\r1-1,1) -- (-\r1+1,1) -- (-\r1+1,-1) -- (\r1-1,-1) -- (\r1-1,1);
  					\filldraw[red,  opacity = 0.5] (1,\r1-1) -- (1,-\r1+1) -- (-1,-\r1+1) -- (-1,\r1-1) -- (1,\r1-1);
  					
  					\foreach \x\y in {-2/1, -1/2, 0.5/0.5, 1.5/1.5, -1.5/-1.5, -0.5/-0.5, 2/-1, 1/-2} {
        					\draw (\x,\y) circle (0.2cm);
        					\node at (\x,\y) {-};
      				}
				\end{tikzpicture}
				\caption{Vision de l'atome de Thomson ou "plum pudding"}
			\end{figure}
			
		\end{column}
	\end{columns}
\end{frame}

\begin{frame}{L'électron}

	\begin{columns}
		\begin{column}{0.7\textwidth}
		
			\begin{block}{Historique}
				\begin{description}
					\item[1838-1851] Prédiction de l'électron par Richard Laming 
					% dans une série d'article
					\item[1897] Découverte de l'électron par Joseph John Thomson
				\end{description}
			\end{block}

			\begin{block}{Propriétés}
				\begin{description}
					\item[1838-1851] Prédiction de l'électron par Richard Laming 
					% dans une série d'article
					\item[1897] Découverte de l'électron par Joseph John Thomson
				\end{description}
			\end{block}

		\end{column}
		\begin{column}{0.25\textwidth}
			
		\end{column}
	\end{columns}
\end{frame}

\subsubsection{Le noyau}
\begin{frame}{Modèle de Rutherford}

	\begin{block}{L'expérience de Rutherford}
%		\begin{columns}
%			\begin{column}
%			htgrgf
%			\end{column}
%			\begin{column}
%			hgvfgfgr
%			\end{column}
%		\end{columns}
	\end{block}

	\begin{block}{Conclusion}
		L'atome est constitué principalement de vide.
	\end{block}

\end{frame}

\subsubsection{Les éléments}
\begin{frame}{Les différents atomes}
	\begin{figure}
		\includegraphics[width=\textwidth]{img/Tableau_périodique_des_éléments.png} 
		\caption{Tableau de Mendeleïev de 2016 avec 118 éléments\cite{element}}
		\label{fig:mendeleiev}
	\end{figure}
\end{frame}

\subsubsection{La représentation}
\begin{frame}{L'évolution de la représentation de l'atome}

\begin{columns}
	\begin{column}{0.3\textwidth}
		\begin{figure}
			\center
			\begin{tikzpicture}[scale=0.5]
				\shade[ball color = gray!40, opacity = 0.4] (0,0) circle (3cm);
  				\draw (0,0) circle (3cm);
  				\draw (-3,0) arc (180:360:3 and 0.6);
  				\draw[dashed] (3,0) arc (0:180:3 and 0.6);
			\end{tikzpicture}
			\caption{Vision de l'atome indivisible}
		\end{figure}
	\end{column}
	
	\begin{column}{0.3\textwidth}
		\begin{figure}
			\begin{tikzpicture}[scale=0.5]
				\shade[ball color = gray!40, opacity = 1] (0,0) circle (1cm);
  				\draw (0,0,0) circle (1cm);			
			
  				\begin{scope}[canvas is zy plane at x=0]
    					\draw[dashed, green, opacity = .5] (0,0) circle (3cm);
    					\node[green] at (2.6,-1.5) {\textbullet};
  				\end{scope}

  				\begin{scope}[canvas is zx plane at y=0]
    					\draw[dashed, red, opacity = .5] (0,0) circle (3cm);
    					\node[red] at (2.1,2.1) {\textbullet};
  				\end{scope}

  				\begin{scope}[canvas is xy plane at z=0]
    					\draw[dashed, blue, opacity = .5] (0,0) circle (3cm);
    					\node[blue] at (-2.1,2.1) {\textbullet};
  				\end{scope}			
			\end{tikzpicture}
			\caption{Vision de l'atome de Rutherford}
		\end{figure}
	\end{column}
	
	\begin{column}{0.4\textwidth}
		\begin{figure}
			\includegraphics[width=\textwidth]{img/atome.png} 
			\caption{Vision de l'atome}
		\end{figure}
	\end{column}
\end{columns}

\end{frame}


\subsection{Le noyau}

\subsubsection{Les nucléons}
\begin{frame}{Les nucléons}

\end{frame}

\subsubsection{Les quarks}
\begin{frame}{Les quarks}

\end{frame}

\subsection{Les leptons}

\begin{frame}{Les leptons}
	
	\begin{columns}
		\begin{column}{0.45\textwidth}
			\begin{block}{Les leptons chargés}
				\begin{itemize}
					\item 
				\end{itemize}
			\end{block}
		\end{column}
		\begin{column}{0.45\textwidth}
			\begin{block}{Les leptons neutres}
				\begin{itemize}
					\item 
				\end{itemize}
			\end{block}
		\end{column}
	\end{columns}

\end{frame}


\subsection{Bilan des particules}
\begin{frame}{Vers l’infiniment petit}{Bilan des constituants de la matière et des interactions fondamentales}

\end{frame}

\subsection{L’anti-matière}
\begin{frame}{Vers l’infiniment petit}{Les nucléons, Les quarks}

\end{frame}

\subsection{L'origine de la Matière ?}
\begin{frame}{Vers l’infiniment petit}{Le modèle du Big Bang en physique des particules, l'origine de la matière ?}

\end{frame}

%\subsection*{Conclusion}
\begin{frame}{Vers l’infiniment petit}{Conclusion}
	\begin{itemize}
		\item 
		\pause         
	\end{itemize}
\end{frame}

%%%%%%%%%%%%%%%%%%%% L'infiniment grand %%%%%%%%%%%%%%%%%%%%

\section{Vers l’infiniment grand}

\begin{frame}
\tableofcontents[currentsection, hideallsubsections]
\end{frame}

\subsection{Les 3 Premières minutes}
\begin{frame}{Vers l’infiniment grand}{Les 3 premières minutes}

\end{frame}

\subsection{Fond Diffus Cosmologique}
\begin{frame}{Vers l’infiniment grand}{Fond Diffus Cosmologique}

\end{frame}

\subsection{Les premiers trous noirs et les premières étoiles}
\begin{frame}{Vers l’infiniment grand}{Les premiers trous noirs et les premières étoiles}

\end{frame}

\subsection{Les différentes générations d’étoiles, Diagramme HR}

\begin{frame}{Vers l’infiniment grand}{Les différentes générations d’étoiles, Diagramme HR}

\end{frame}

\subsection{Les réactions astronucléaires, la formation des métaux}
\begin{frame}{Vers l’infiniment grand}{Les réactions astronucléaires, la formation des métaux}

\end{frame}

\subsection{Bilan sur le modèle Standard de la Cosmologie}
\begin{frame}{Vers l’infiniment grand}{Bilan sur le modèle Standard de la Cosmologie}

\end{frame}

%%%%%%%%%%%%%%%%%%%% L'au-delà %%%%%%%%%%%%%%%%%%%%

\section{Vers l’au-delà}

\begin{frame}
\tableofcontents[currentsection, hideallsubsections]
\end{frame}

\subsection{Problème de la Matière manquant}
\begin{frame}{Vers l’au-delà}{Problème de la Matière manquant}{La matière baryonique : visible \& invisible}

\end{frame}

\begin{frame}{Vers l’au-delà}{Problème de la Matière manquant}{Matière noire ?}

\end{frame}

\begin{frame}{Vers l’au-delà}{Problème de la Matière manquant}{Gravité modifiée ?}

\end{frame}

\begin{frame}{Vers l’au-delà}{Problème de la Matière manquant}{Hypothèses supplémentaires}

\end{frame}

\subsection{Problème de l’expansion accélérée}
\begin{frame}{Vers l’au-delà}{Problème de l’expansion accélérée}{Énergie noire, Vide ?}

\end{frame}

\subsection{Problème de la relativité quantique}
\begin{frame}{Vers l’au-delà}{Problème de la relativité quantique}

\end{frame}

%%%%%%%%%%%%%%%%%%%% Conclusion %%%%%%%%%%%%%%%%%%%%

\section*{Conclusion}

\begin{frame}
\tableofcontents[currentsection, hideallsubsections]
\end{frame}

\begin{frame}{Conclusion}{Bilan de la présentation}

\end{frame}

\begin{frame}{Conclusion}{Limites de cette présentation}

\end{frame}

\begin{frame}{Conclusion}{État de la Recherche dans ce domaine}

\end{frame}

\begin{frame}{Conclusion}
Merci de votre attention !
\end{frame}

% Sommaire
\begin{frame}{Sommaire}
	\begin{tiny}
	\begin{columns}[t]
		\begin{column}{0.3\textwidth}
			\tableofcontents[sections=1]
		\end{column}
		\begin{column}{0.3\textwidth}
			\tableofcontents[sections=2]
		\end{column}
		\begin{column}{0.3\textwidth}
			\tableofcontents[sections=3]
		\end{column}
	\end{columns}
	\end{tiny}
\end{frame}

%%%%%%%%%%%%%%%%%%%% Références %%%%%%%%%%%%%%%%%%%%

\section*{Références}
\begin{frame}[allowframebreaks]{References} 
	\nocite{*}
	\bibliographystyle{unsrt} 
	\bibliography{biblio}
\end{frame}

%%%%%%%%%%%%%%%%%%%% Annexes %%%%%%%%%%%%%%%%%%%%

\section*{Annexes}
\begin{frame}{Annexes}


\end{frame}

\end{document}