\documentclass[handout,8pt]{beamer} % handout : désactive les pauses
%betel arnault 

\usetheme[compress]{Dresden}
%\setbeamertemplate{page number in head/foot}[framenumber]
\addtobeamertemplate{navigation symbols}{}{ 
	\hspace{1em}    
	\usebeamerfont{footline}%
    \insertframenumber / \inserttotalframenumber 
}
    
\usepackage[utf8]{inputenc}
\usepackage[french]{babel}
\usepackage[T1]{fontenc}
\uselanguage{French}
\languagepath{French}

\usepackage{amsmath}
\usepackage{amsfonts}
\usepackage{amssymb}

\usepackage{graphicx}
\usepackage{hyperref}
\usepackage{multicol}
\usepackage{tikz}
\usetikzlibrary{shapes.geometric}
\usetikzlibrary{3d}



\usepackage{siunitx}
\sisetup{
    detect-all,
    output-decimal-marker={,},
    group-minimum-digits = 3,
    group-separator={~},
    number-unit-separator={~},
    inter-unit-product={~}
}

\newcommand{\backupbegin}{
   \newcounter{framenumberappendix}
   \setcounter{framenumberappendix}{\value{framenumber}}
}
\newcommand{\backupend}{
   \addtocounter{framenumberappendix}{-\value{framenumber}}
   \addtocounter{framenumber}{\value{framenumberappendix}} 
}

%%%%%%%%%%%%%%%%%%%% Page de garde %%%%%%%%%%%%%%%%%%%%

\title{La Physique des 2 Infinis}
\author[A. Hocine]{Alexia HOCINE\\ \textit{Physicienne subatomique \& Développeuse}}

%\setbeamercovered{transparent} 
%\setbeamertemplate{navigation symbols}{} 
%\logo{} 

\institute[LVA]{Les Vendredis de l'Astronomie} 

\date[2023]{Vendredi 1\ier\ Décembre 2023} 

%%%%%%%%%%%%%%%%%%%% Environnement %%%%%%%%%%%%%%%%%%%%

% definition

\begin{document}

%%%%%%%%%%%%%%%%%%%% Page de garde %%%%%%%%%%%%%%%%%%%%

\begin{frame}
	\titlepage
\end{frame}

%%%%%%%%%%%%%%%%%%%% Introduction %%%%%%%%%%%%%%%%%%%%

% Présentation
\begin{frame}{Introduction}{Présentation de mon Parcourt}

	\pause
	\begin{columns}
		\begin{column}{0.45\textwidth}
			\begin{block}{Licence Informatique}
				\begin{itemize}
					\item Université de Rouen-Normandie
					\item Stage chez SIQUAL
					\begin{itemize}
						\item Site et Application Web
						\item Construction de Base de Données
					\end{itemize}
				\end{itemize}
			\end{block}
		\end{column}
		\pause
		\begin{column}{0.45\textwidth}
			\begin{block}{Licence Physique}
				\begin{itemize}
					\item Université de Rouen-Normandie
					\item Stage au GPM (Groupe de Physique des Matériaux)
					\begin{itemize}
						\item Information Quantique
						\item État des lieux de la Recherche actuelle
					\end{itemize}
				\end{itemize}
			\end{block}
		\end{column}
	\end{columns}
	\pause
	\begin{block}{Master Physique Subatomique}
		\begin{itemize}
			\item UCBL (Université de Claude Bernard Lyon 1)
		\end{itemize}
		\begin{columns}
			\begin{column}{0.45\textwidth}
				\begin{itemize}
					%\item Physique des Particules,
					 %		Relativité,
					%		Physique Statistique
					\item Stage au CRAL (Centre de Recherche d'Astrophysique de Lyon)
					\begin{itemize}
						\item Influence du comportement du disque protoplanétaire avec une planète formée rapidement
						\item Interprétation de simulations numériques
					\end{itemize}
				\end{itemize}
			\end{column}
			\pause
			\begin{column}{0.45\textwidth}
				\begin{itemize}
					%\item Physique Statistique avancée,
							%Cosmologie Observationnelle,
							%Ondes Gravitationnelles
					\item Stage à l'IP2I (Institut de Physique des 2 Infinies)
					\begin{itemize}
						\item Équipe CMS, FCC (CERN)
						\item Construction de Simulation numérique à la Recherche du Boson de Higgs
					\end{itemize}
				\end{itemize}
			\end{column}
		\end{columns}
	\end{block}
\end{frame}

% Problématiques
\begin{frame}{Introduction}{Objectifs de cette conférence}
	\begin{center}
		\begin{Large}
			De quoi est constitué l'Univers ?
		\end{Large}
	\end{center}
	
	\begin{enumerate}
		\pause
		\item Quelles sont les constituants élémentaires ? Quelles sont leurs interactions ?
		\pause
		\item Quand ? Où ? Comment se sont-ils formés ?
		\pause
		\item Les connaissons-nous tous ? 
	\end{enumerate}
\end{frame}

% Plan général
\begin{frame}{Introduction}{Sommaire}
	\begin{center}
		\begin{Large}
			\textbf{La Physique des 2 Infinis}
		\end{Large}
	\end{center}

	\tableofcontents[hideallsubsections] % pausesections
\end{frame}

%%%%%%%%%%%%%%%%%%%% L'infiniment petit %%%%%%%%%%%%%%%%%%%%

\section{Vers l’infiniment petit}
\begin{frame}{Vers l’infiniment petit}{Sommaire}
	\begin{center}
		\begin{Large}
			\textbf{La Physique des 2 Infinis}
		\end{Large}
	\end{center}
    \tableofcontents[currentsection, hideallsubsections]
\end{frame}

\subsection{La Matière}
\begin{frame}{Qu'est-ce que la Matière ?}

\begin{definition}[La Matière \cite{matter}]
	\begin{itemize}
		\item ce qui compose tout corps,
		\item objet qui occupe de l'espace et qui ont une masse. 
	\end{itemize}
	Réciproquement, tout ce qui a une masse est de la matière.
\end{definition}
	
	\pause
	\begin{center}
		\textbf{De quoi est composé la matière ?}
	\end{center}
\end{frame}

\subsubsection{L’Atome}
\begin{frame}{De quoi est composée la Matière ?}

\begin{columns}
	\begin{column}{0.7\textwidth}
	
		\begin{block}{Intuition de Démocrite (460-370 avant J.-C.)\cite{democrite}}
			La matière est composée d'objets élémentaires \textit{indivisibles}. 
		\end{block}
		\pause

		\begin{block}{Aristote, Nouveau Testament, ... }
			\begin{itemize}
				\item Nommé "\textbf{atome}" (partie de matière indivisible)
			\end{itemize}
		\end{block}

		\begin{definition}[L'Atome\cite{atom}]
			\begin{itemize}
				\item Petit corps
				\item La plus petite partie d'un corps simple pouvant se combiner chimiquement avec un autre. 
				\item Les constituants élémentaires de toutes les substances solides, liquides ou gazeuses.
			\end{itemize}
		\end{definition}

	\end{column}
	
	\begin{column}{0.25\textwidth}
		\begin{figure}
			\center
			\begin{tikzpicture}[scale=0.5]
				\shade[ball color = gray!40, opacity = 0.4] (0,0) circle (2cm);
  				\draw (0,0) circle (2cm);
  				\draw (-2,0) arc (180:360:2 and 0.6);
  				\draw[dashed] (2,0) arc (0:180:2 and 0.6);
			\end{tikzpicture}
			\caption{1\iere\ vision de l'atome}
		\end{figure}
		
		\begin{figure}
			\begin{tikzpicture}[scale=0.5]
				\shade[ball color = red!80, opacity = 0.4] (0,0) circle (1cm);
  				\draw (0,0) circle (1cm);
  				
  				\shade[ball color = gray!80, opacity = 0.4] (1.7,-1.7) circle (0.5cm);
  				\draw (1.7,-1.7) circle (0.5cm);
  				%\draw[dashed] (0,0) -- (2,-2);
  				\draw (0.7,-0.7) -- (1.3,-1.3);
  				
  				\shade[ball color = gray!80, opacity = 0.4] (-1.7,-1.7) circle (0.5cm);
  				\draw (-1.7,-1.7) circle (0.5cm);
				\draw(-0.7,-0.7) -- (-1.3,-1.3);
			\end{tikzpicture}
			\caption{Représentation d'une molécule d'eau}
		\end{figure}
	\end{column}
\end{columns}

\end{frame}

\subsubsection{L'Électron}

\begin{frame}{L'Électron}
		
	\begin{block}{Historique}
		\begin{description}
			\item[1838-51] Prédiction de l'électron par Richard Laming :
			\begin{itemize}
				\item Afin d'expliquer les propriétés chimiques des atomes
			\end{itemize}
			\item[1894] Stoney pose le mot "électron" :
			\begin{itemize}
				\item  "électrique" + "-on" suffixe des particules subatomiques
			\end{itemize}
			\item[1897] Découverte de l'électron par Joseph John Thomson
			\begin{itemize}
				\item  1\iere\ particule élémentaire découverte
			\end{itemize}
		\end{description}
	\end{block}

	\begin{columns}
		\begin{column}{0.5\textwidth}
			\begin{block}{Description}
				\begin{itemize}
					\item Particule élémentaire
					\item Charge électrique unitaire négative
				\end{itemize}
			\end{block}
			\begin{block}{Rôle dans l'atome}
				\begin{itemize}
					\item Appartient à l'atome
					\item Permet la liaison entre les molécules
				\end{itemize}
			\end{block}
		\end{column}
		\begin{column}{0.5\textwidth}
			\begin{block}{Propriétés}
				\begin{description}
					\item[Masse] $ \SI{9.109e-31}{\kilogram} $ 
								($ = \SI{551}{\keV\per\square c}$)
					\item[Charge] $\SI{-1}{\eV}$ 
								($ = \SI{-1.602e-19}{\coulomb}$)
					\item[Durée de Vie] Stable
					\item[Dimension] Ponctuelle
				\end{description}
			\end{block}
		\end{column}
	\end{columns}
	
\end{frame}

\subsubsection{Les Modèles atomiques}

\begin{frame}{Le Modèle atomique de Thomson}{Le Modèle du \textit{cookie aux pépites de chocolat}}

	\begin{columns}
		\begin{column}{0.7\textwidth}
		
			\begin{block}{Historique}
				\begin{description}
					\item[1838-1851] Prédiction de l'électron par Richard Laming 
					% dans une série d'article
					\item[1897] Découverte de l'électron par Joseph John Thomson
				\end{description}
			\end{block}

			\begin{block}{Modèle atomique de Thomson}
				\begin{description}
					\item[1904] L'atome composé de charges positives et de particules négatives.
					\item[1909] Invalidation du modèle par Rutherford
				\end{description}
			\end{block}

		\end{column}
		\begin{column}{0.25\textwidth}
		
			\begin{figure}
				\center
				\begin{tikzpicture}[scale=0.4]
					\def\r1{4};
					\shade[ball color = red, opacity = 0.5] (0,0) circle (\r1);
  					\draw (0,0) circle (\r1);
  					\draw (-\r1,0) arc (180:360:\r1 and 0.6);
  					\draw[dashed] (\r1,0) arc (0:180:\r1 and 0.6);
  					\filldraw[red,  opacity = 0.5] (\r1-1,1) -- (-\r1+1,1) -- (-\r1+1,-1) -- (\r1-1,-1) -- (\r1-1,1);
  					\filldraw[red,  opacity = 0.5] (1,\r1-1) -- (1,-\r1+1) -- (-1,-\r1+1) -- (-1,\r1-1) -- (1,\r1-1);
  					
  					\foreach \x\y in {-2/1, -1/2, 0.5/0.5, 1.5/1.5, -1.5/-1.5, -0.5/-0.5, 2/-1, 1/-2} {
        					\draw (\x,\y) circle (0.2cm);
        					\node at (\x,\y) {-};
      				}
				\end{tikzpicture}
				\caption{Vision de l'atome de Thomson}
			\end{figure}
			
		\end{column}
	\end{columns}
\end{frame}

\begin{frame}{L'expérience de Rutherford (1909)}{L'expérience de la feuille d'Or}
	
	\begin{figure}
		\center
%		\begin{tikzpicture}
%			\def\xalpha{-1}
%			\def\xor{3};
%			\def\xzinc{6};	
%			\def\h{1}	
%		
%			\coordinate (alpha) at (\xalpha,0,0);
%			\coordinate (O) at (0,0,0);
%			\coordinate (or) at (\xor,0,0);
%			\coordinate (zinc) at (\xzinc,0,0);
%			
%			\draw[gray!20] (-3,-\h*2) grid (\xzinc+1, 3);	
%			
%			\node at (O)[above] {Source de};			
%			\node at (O)[below] {Particules $\alpha$};
%			\node at (\xor,0,\h)[below] {Feuille d'Or};
%			\node at (\xzinc,0,\h)[below] {Écran de sulfure de Zinc};
%			
%			\draw[->] (alpha) -- (or);
%			\fill[yellow!50, opacity=0.7] (\xor,\h,\h) -- (\xor,\h,-\h) -- (\xor,-\h,-\h) -- (\xor,-\h,\h) -- cycle;
%			
%			\draw[->] (or) -- (zinc);				
%			\fill[gray!50, opacity=0.7] (\xzinc,\h,\h) -- (\xzinc,\h,-\h) -- (\xzinc,-\h,-\h) -- (\xzinc,-\h,\h) -- cycle;
%			
%			%\draw (10:-1) arc (10:350:-\rzinc/2);
%			%\draw[gray!50, opacity=0.7] (\xor,0,\h) circle (\xzinc/2);		
%			%\draw[gray!50, opacity=0.7] (\xor,0,-\h) circle (\xzinc/2);	
%			\node[cylinder, 
%    				draw = gray, 
%    				cylinder uses custom fill, 
%    				cylinder body fill = gray!50, 
%    				cylinder end fill = gray!70,
%    				inner xsep = \h cm, 
%    				minimum width = \xzinc / 2 cm,
%    				aspect = 0.2, 
%    				shape border rotate = 90] (c) at (or) {};
%
%		\end{tikzpicture}
		\includegraphics[width=0.5\textwidth]{img/ExpRutherford.jpg} 
		\caption{Expérience de Rutherford}
	\end{figure}

	\begin{enumerate}
		\item Émission de particules $\alpha$ (noyau d'Hélium : $^4_2He^{2+}$)
		\item Une feuille d'Or de $\SI{6000}{\angstrom}$ ($\si{\angstrom} = \SI{e-10}{\meter}$)
		\item Un écran de Sulfure de Zinc (ZnS) 
		\begin{itemize}
			\item Lors d'une collision ($\alpha + ZnS$), on observe un scintillement lumineux
		\end{itemize}
	\end{enumerate}
,\end{frame}

\begin{frame}{Modèle de Rutherford (1909)}

	\begin{block}{Résultat de l'expérience de Rutherford}
		$\approx 99,99\%$ des particules $\alpha$ ne sont pas déviées
		\begin{itemize}
			\item donc l'atome est principalement constitué de vide
		\end{itemize}
	\end{block}

	\begin{columns}
		\begin{column}{0.45\textwidth}
			\begin{figure}
				\begin{tikzpicture}
					\draw[black,fill=red!20] (0,0) circle (1);
					\foreach \a in {20,80,..., 360} {
						\draw[blue,fill=blue!20] (\a:0.7) circle (0.05);
					}
					\draw[blue,fill=blue!20] (0,0) circle (0.05);
					\foreach \k in {-0.9,-0.6,...,0.9} {
						\draw[->] (-1.2, \k) -- (1.2, \k);
					}
				\end{tikzpicture}
			\caption{Modèle de Thomson}
			\end{figure}
		\end{column}
		\begin{column}{0.45\textwidth}
			\begin{figure}
				\begin{tikzpicture}
					\draw[black, dashed] (0,0) circle (1);
					\foreach \a in {20,80,..., 360} {
						\draw[blue,fill=blue!20] (\a:0.7) circle (0.05);
					}
					\draw[red,fill=red!20] (0,0) circle (0.15);
					\foreach \k in {-0.9,-0.6,-0.3, 0.3, 0.6, 0.9} {
						\draw[->] (-1.2, \k) -- (1.2, \k);
					}
					\draw[->] (-1.2, 0) -- (-0.2, 0) -- (-1.1,1.1);
				\end{tikzpicture}
			\caption{Modèle de Rutherford}
			\end{figure}
		\end{column}
	\end{columns}
	
	\begin{block}{Modèle de Rutherford, Modèle planétaire}
		\begin{description}
			\begin{columns}
					\begin{column}{0.35\textwidth}
					\item[Noyau]
			\begin{itemize}
				\item de charge positive
				\item très petit
				\item au centre
			\end{itemize}
			
					\end{column}
					\begin{column}{0.35\textwidth}
					\item[Électron]
			\begin{itemize}
				\item de charge négative
				\item qui "gravitent" autour du noyau
			\end{itemize}
					\end{column}
				\end{columns}
		\end{description}
		                                                                                                                                                                                                                                                                                                                                                                                                                                                                                                                                                                                                                                                                                                                                                                                                                                                                                                                                                                                                                                                                                                                                                                                                                                                                                                                                                                                                                                                                                                                                                                                                                                                                                                                                                                                                                                                                                                                                                                                                                                                                                                                                                                                                                                                                                                                                                                                                                                                                                                                                                                                                                                                                                                                                                                                                                                                                                                                                                                                                                                                                                                                                                                                                                                                                                                                                                                                                                                                                                                                                                                                                                                                                                                                                                                                                                                                                                                                                                                                                                                                                                                                                                                                                                                                                                                                                                                                                                                                                                                                                                                                                                                                                                                                                                                                                                                                                                                                                                                                                                                                                                                                                                                                                                                                                                                                                                                                                                                                                                                                                                                                                                                                                                                                                                                                                                                                                                                                                                                                                                                                                                                                                                                                                                                                                                                                                                                                                                                                                                                                                                                                                                                                                                                                                                                                                                                                                                                                                                                                                                                                                                                                                                                                                                                                                                                                                                                                                                                                                                                                                                                                                                                                                                                                                                                                                                                                                                                                                                                                                                                                                                                                                                                                                                                                                                                                                                                                                                                                                                                                                                                                                                                                                                                                                                                                                                                                                                                                                                                                                                                                                                                                                                                                                                                                                 
	\end{block}

\end{frame}

\begin{frame}{Modèle de Schrödinger (1925)}

	\begin{columns}
		\begin{column}{0.45\textwidth}
			\begin{block}{Électron}
				\begin{itemize}
					\item Objet quantique, dualité onde-corpuscule
					\item pas de localisation précise, mais une probabilité de présence
				\end{itemize}
			\end{block}
		\end{column}
		\begin{column}{0.45\textwidth}
			\begin{figure}
			\includegraphics[width=\textwidth]{img/fo_H.png} 
			\caption{en fonction de 3 nb quantiques : l'énergie de l'électron, son moment angulaire et la projection de ce moment angulaire sur un axe donné.}
			\end{figure}
		\end{column}
	\end{columns}

\end{frame}

\subsubsection{Le Noyau atomique}

\begin{frame}{Le Noyau}

	\begin{columns}
		\begin{column}{0.45\textwidth}
			\begin{block}{Les Nucléons}
				\begin{itemize}
				\item particule subatomique \textbf{non} élémentaire
				\item de masse 
					$\approx \SI{1,674e-27}{\kg}$ 
					$\approx \SI{939,5}{\MeV\per\square c}$ 
				\item $A$ le nombre de masse (ou de nucléon)
		\end{itemize}
			\end{block}
		\end{column}
		\begin{column}{0.45\textwidth}
			\begin{figure}
			\begin{tikzpicture}
				\draw[red, dashed,fill=red!20] (0,0) circle (1);
				\foreach \a in {10,70,..., 360} {
					\draw[blue,fill=blue!20] (\a:0.8) circle (0.1);
					\draw[green,fill=green!20] (\a+30:0.8) circle (0.1);
				}
				\foreach \a in {10,130, ..., 360} {
					\draw[blue,fill=blue!20] (\a+60:0.4) circle (0.1);
					\draw[green,fill=green!20] (\a:0.4) circle (0.1);
				}
				\draw[green,fill=green!20] (0:0) circle (0.1);
			\end{tikzpicture}
			\caption{Représentation schématique d'un noyau}
			\end{figure}
		\end{column}
	\end{columns}
	
	\begin{columns}
		\begin{column}{0.45\textwidth}
			\begin{block}{Proton (nucléon)}
				\begin{itemize}
					\item de charge électrique positive
					\item détermine la nature de l'élément chimique 
				\end{itemize}
				\begin{description}
					\item[Masse] $\SI{938,272}{\MeV \per\square c}$
					\item[Charge] $+e = \SI{1}{\eV} = \SI{1,602176565e-19}{\coulomb}$
					\item[Durée de vie] Stable
				\end{description}
			\end{block}
		\end{column}
		\begin{column}{0.45\textwidth}
			\begin{block}{Neutron (nucléon)}
				\begin{itemize}
					\item sans charge électrique
					\item détermine l'isotope 
				\end{itemize}
				\begin{description}
					\item[Masse] $\SI{939,5654}{\MeV \per\square c}$
					\item[Charge] $\SI{0}{}$
					\item[Durée de vie] $\SI{880,3}{\second}$ mais stable dans un noyau
				\end{description}
			\end{block}
		\end{column}
	\end{columns}		
\end{frame}

\begin{frame}{Les différents atomes}
	\begin{figure}
		\includegraphics[width=0.9\textwidth]{img/Tableau_périodique_des_éléments.png} 
		\caption{Tableau de Mendeleïev de 2016 avec 118 éléments\cite{element}}
		\label{fig:mendeleiev}
	\end{figure}
\end{frame}

\subsubsection{Les Nucléons}
\begin{frame}{Les Nucléons}

	\begin{block}{Nucléon}
		\begin{itemize}
			\item Baryon : particule composé de 3 quarks
			\begin{itemize}
				\item particule élémentaire 
			\end{itemize}
		\end{itemize}
	\end{block}
	
	\begin{columns}
		\begin{column}{0.45\textwidth}
			\begin{block}{Proton}
				\begin{itemize}
					\item 2 quarks \textbf{u}
					\item 1 quark \textbf{d}
				\end{itemize}
			\end{block}
			\begin{figure}
				\begin{tikzpicture}
					\draw[gray,fill=gray!20] (0,0) circle (1);
					\draw[cyan,fill=cyan!20] (60:0.5) circle (0.25) node{u};
					\draw[cyan,fill=cyan!20] (180:0.5) circle (0.25)node{u};
					\draw[violet,fill=violet!20] (300:0.5) circle (0.25) node{d};;
				\end{tikzpicture}
				\caption{Proton}
			\end{figure}
		\end{column}
		\begin{column}{0.45\textwidth}
			\begin{block}{Proton}
				\begin{itemize}
					\item 1 quark \textbf{u}
					\item 2 quarks \textbf{d}
				\end{itemize}
			\end{block}
			\begin{figure}
				\begin{tikzpicture}
					\draw[gray,fill=gray!20] (0,0) circle (1);
					\draw[cyan,fill=cyan!20] (60:0.5) circle (0.25) node{u};
					\draw[violet,fill=violet!20] (180:0.5) circle (0.25)node{d};
					\draw[violet,fill=violet!20] (300:0.5) circle (0.25) node{d};;
				\end{tikzpicture}
				\caption{Neutron}
			\end{figure}
		\end{column}
	\end{columns}		
\end{frame}

\subsubsection{Bilan sur la composition de la matière}
\begin{frame}{Bilan de la composition de la Matière}

	\begin{figure}
		\includegraphics[width=\textwidth]{img/resume_matiere.png} 
		\caption{Décomposition de la Matière}
	\end{figure}

\end{frame}

\subsection{La Matière hadronique}
\begin{frame}{La Matière hadronique}

\end{frame}

\subsubsection{Les Quarks}
\begin{frame}{Les Quarks}
	\begin{block}{Historique}
		\begin{description}
			\item[1964] Hypothèse de Murray Gell-Mann et George Zweig
			\item[1968] Découverte des 1\iers\ quarks \textbf{u}, \textbf{b} et \textbf{s}
			\item[1969] Prix Nobel pour Murray Gell-Mann
		\end{description}
	\end{block}

	\begin{itemize}
		\item particule élémentaire
		\item 6 saveurs :
		\begin{tabular}{ c c c c c c c c c }
			& Nom & Masse [en $\si{\GeV\per\square c}$] & Prédiction & Découverte \\
			\textbf{u} & up & $0,003$ & 1964 & 1968 \\
			\textbf{d} & down & $0,006$ & 1964 & 1968 \\
			\textbf{s} & strange & $0,1$ & 1964 & 1968 \\
			\textbf{c} & charm & $1,3$ & 1970 & 1974 \\
			\textbf{b} & bottom & $4,3$ & 1973 & 1977 \\
			\textbf{t} & top & $175$ & 1973 & 1995 \\
		\end{tabular}
		\item particules confinés dans des \textbf{hadrons}
		\item \textit{le quark \textbf{top} est le seul observable !}
	\end{itemize}
\end{frame}

\subsubsection{Les Leptons}

\begin{frame}{Les leptons}
	
	\begin{itemize}
		\item particules élémentaires
		\item 6 saveurs
	\end{itemize}		
	
	\begin{block}{Les leptons chargés}
		\begin{itemize}
			\item 3 saveurs :
			\begin{tabular}{ c c c c c c c c c }
				& Nom & Masse [en $\si{\GeV\per\square c}$] & Prédiction & Découverte \\
				\textbf{$e^-$} & électron & $0,000511$ & 1874 & 1897 \\
				\textbf{$\mu^-$} & muon & $0,106$ & $\times$ & 1936 \\
				\textbf{$\tau^-$} & tau & $1,7771$ & $\times$ & 1975 \\
			\end{tabular}
			\item seul l'électron est stable
		\end{itemize}
	\end{block}

	\begin{block}{Les leptons neutres ou neutrinos}
		\begin{itemize}
			\item 3 saveurs :
			\begin{tabular}{ c c c c c c c c c }
				& Nom & Masse [en $\si{\GeV\per\square c}$] & Prédiction & Découverte \\
				\textbf{$\nu_e$} & électron & $< \num{2.5e-9}$ & 1930 & 1956 \\
				\textbf{$\nu_\mu$} & muon & $< \num{0.17e-6}$ & 1940s & 1962 \\
				\textbf{$\nu_\tau$} & tau & $< \num{18e-3}$ & 1970s & 2000 \\
			\end{tabular}
			\item masse très faible mais non nulles 
			\item stable (\textit{mais oscille})
		\end{itemize}
	\end{block}

\end{frame}



\subsection{Les Interactions}

\begin{frame}{Interaction Électromagnétique}

	\begin{block}{}
		\begin{itemize}
			\item Charge électrique
			\begin{itemize}
				\item Les signes opposés s'attirent 
				\item Les signes identiques se repoussent
			\end{itemize}
			\item Portée infinie 
			\item Force additive
		\end{itemize}
	\end{block}
\end{frame}

\begin{frame}{Interaction Faible}
	
\end{frame}

\begin{frame}{Interaction Forte}
	
\end{frame}

\begin{frame}{Interaction Gravitationnelle}
	
\end{frame}

\subsection{Physique des particules}
\begin{frame}{Bilan des constituants de la matière et des interactions fondamentales}

\end{frame}

\subsection{L’anti-matière}
\begin{frame}{L'anti-matière}

\end{frame}

%%%%%%%%%%%%%%%%%%%% L'infiniment grand %%%%%%%%%%%%%%%%%%%%

\section{Vers l’infiniment grand}

\begin{frame}
\tableofcontents[currentsection, hideallsubsections]
\end{frame}

\subsection{L'origine de la Matière ?}
\begin{frame}{Le modèle du Big Bang en physique des particules, l'origine de la matière ?}

\end{frame}

\subsection{Les 3 Premières minutes}
\begin{frame}{Vers l’infiniment grand}{Les 3 premières minutes}

\end{frame}

\subsection{Fond Diffus Cosmologique}
\begin{frame}{Vers l’infiniment grand}{Fond Diffus Cosmologique}

\end{frame}

\subsection{Les premiers trous noirs et les premières étoiles}
\begin{frame}{Vers l’infiniment grand}{Les premiers trous noirs et les premières étoiles}

\end{frame}

\subsection{Les différentes générations d’étoiles, Diagramme HR}

\begin{frame}{Vers l’infiniment grand}{Les différentes générations d’étoiles, Diagramme HR}

\end{frame}

\subsection{Les réactions astronucléaires, la formation des métaux}
\begin{frame}{Vers l’infiniment grand}{Les réactions astronucléaires, la formation des métaux}

\end{frame}

\subsection{Bilan sur le modèle Standard de la Cosmologie}
\begin{frame}{Vers l’infiniment grand}{Bilan sur le modèle Standard de la Cosmologie}

\end{frame}

%%%%%%%%%%%%%%%%%%%% L'au-delà %%%%%%%%%%%%%%%%%%%%

\section{Vers l’au-delà}

\begin{frame}
\tableofcontents[currentsection, hideallsubsections]
\end{frame}

\subsection{Problème de la Matière manquant}
\begin{frame}{Vers l’au-delà}{Problème de la Matière manquant}{La matière baryonique : visible \& invisible}

\end{frame}

\begin{frame}{Vers l’au-delà}{Problème de la Matière manquant}{Matière noire ?}

\end{frame}

\begin{frame}{Vers l’au-delà}{Problème de la Matière manquant}{Gravité modifiée ?}

\end{frame}

\begin{frame}{Vers l’au-delà}{Problème de la Matière manquant}{Hypothèses supplémentaires}

\end{frame}

\subsection{Problème de l’expansion accélérée}
\begin{frame}{Vers l’au-delà}{Problème de l’expansion accélérée}{Énergie noire, Vide ?}

\end{frame}

\subsection{Problème de la relativité quantique}
\begin{frame}{Vers l’au-delà}{Problème de la relativité quantique}

\end{frame}

%%%%%%%%%%%%%%%%%%%% Conclusion %%%%%%%%%%%%%%%%%%%%

\section*{Conclusion}

\begin{frame}
\tableofcontents[currentsection, hideallsubsections]
\end{frame}

\begin{frame}{Conclusion}{Bilan de la présentation}

\end{frame}

\begin{frame}{Conclusion}{Limites de cette présentation}

\end{frame}

\begin{frame}{Conclusion}{État de la Recherche dans ce domaine}

\end{frame}

\begin{frame}{Conclusion}
Merci de votre attention !
\end{frame}

% Sommaire
\begin{frame}{Sommaire}
	\begin{tiny}
	\begin{columns}[t]
		\begin{column}{0.3\textwidth}
			\tableofcontents[sections=1]
		\end{column}
		\begin{column}{0.3\textwidth}
			\tableofcontents[sections=2]
		\end{column}
		\begin{column}{0.3\textwidth}
			\tableofcontents[sections=3]
		\end{column}
	\end{columns}
	\end{tiny}
\end{frame}

%%%%%%%%%%%%%%%%%%%% Références %%%%%%%%%%%%%%%%%%%%

\subsection*{Références}
\begin{frame}[allowframebreaks]{References} 
	\nocite{*}
	\bibliographystyle{unsrt} 
	\bibliography{biblio}
\end{frame}

%%%%%%%%%%%%%%%%%%%% Annexes %%%%%%%%%%%%%%%%%%%%
\appendix
\backupbegin

\section*{Annexes}
\begin{frame}{Annexes}
\end{frame}

\subsubsection{L'évolution de la représentation de l'atome}
\begin{frame}{L'évolution de la représentation de l'atome}

\begin{columns}
	\begin{column}{0.3\textwidth}
		\begin{figure}
			\center
			\begin{tikzpicture}[scale=0.5]
				\shade[ball color = gray!40, opacity = 0.4] (0,0) circle (3cm);
  				\draw (0,0) circle (3cm);
  				\draw (-3,0) arc (180:360:3 and 0.6);
  				\draw[dashed] (3,0) arc (0:180:3 and 0.6);
			\end{tikzpicture}
			\caption{Vision de l'atome indivisible}
		\end{figure}
	\end{column}
	
	\begin{column}{0.3\textwidth}
		\begin{figure}
			\begin{tikzpicture}[scale=0.5]
				\shade[ball color = gray!40, opacity = 1] (0,0) circle (1cm);
  				\draw (0,0,0) circle (1cm);			
			
  				\begin{scope}[canvas is zy plane at x=0]
    					\draw[dashed, green, opacity = .5] (0,0) circle (3cm);
    					\node[green] at (2.6,-1.5) {\textbullet};
  				\end{scope}

  				\begin{scope}[canvas is zx plane at y=0]
    					\draw[dashed, red, opacity = .5] (0,0) circle (3cm);
    					\node[red] at (2.1,2.1) {\textbullet};
  				\end{scope}

  				\begin{scope}[canvas is xy plane at z=0]
    					\draw[dashed, blue, opacity = .5] (0,0) circle (3cm);
    					\node[blue] at (-2.1,2.1) {\textbullet};
  				\end{scope}			
			\end{tikzpicture}
			\caption{Vision de l'atome de Rutherford}
		\end{figure}
	\end{column}
	
	\begin{column}{0.4\textwidth}
		\begin{figure}
			\includegraphics[width=\textwidth]{img/atome.png} 
			\caption{Vision de l'atome}
		\end{figure}
	\end{column}
\end{columns}

\end{frame}


\backupend

\end{document}
